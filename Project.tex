\documentclass[pdf]{beamer}
\mode<presentation>{\usetheme{Warsaw}}
\usepackage{array}
\title{Linear Algebra}
\subtitle{Chapter 3}
\author{To Thai Duong}
\institute{ICT - 01 K65}
\date{January 2021}
\AtBeginSection[]
{
\begin{frame}{Overview}
\tableofcontents[currentsection]
\end{frame}
}
\begin{document}
\begin{frame}
\titlepage
\end{frame}
\begin{frame}{Overview}
\tableofcontents
\end{frame}
\section{Vector space}
\begin{frame}{Concepts}
\setlength{\textwidth}{11.2cm}
    \begin{block}{Definition}
    Let $V$ be non-empty set and $\mathbb{K}$ be a field. Two operations are defined as follows.
    \begin{enumerate}
        \item[i)] The first operation, called vector addition or simply additon
        \begin{displaymath}
        \begin{array}
                     "+" : V x V \rightarrow V\\
                    (u,v) \rightarrow u+v   
        \end{array}
        \end{displaymath}
        \item[ii)] The second operation, called scalar multiplication
        \begin{displaymath}
        \begin{array} 
                     "." : K x V \rightarrow V\\
                    (k,v) \rightarrow kv   
        \end{array}
        \end{displaymath}
    \end{enumerate}
    \end{block}
\end{frame}
\begin{frame}{Concepts}
    \setlength{\textwidth}{11.2cm}
    \begin{block}{Definition}
    Now, $(V;+;.)$ is called a vector space over $\mathbb{K}$ if eight axioms hold
    \begin{table}
        \begin{tabular}{|m{3.2cm}|m{7cm}|}
            \hline
            Axiom & Meaning\\ 
            \hline
            1) Associativity of addition & $u+(v+w)=(u+v)+w,u,v,w \in V$\\
            \hline
            2) Commutativity of addition & $u+v=v+u,u,v \in V$\\
            \hline
            3) Identity  element  of  addition & There exists an element $0 \in V$, called the zero vector, such that $v+ 0 =v,\forall v \in V$\\
            \hline
            4) Inverse elements of addition & For every $v \in V$, there exists an element $−v \in V$, called the additive inverse of v,such that $v+ (−v) = 0$\\
            \hline
        \end{tabular}
    \end{table}
    \end{block}
\end{frame}
\begin{frame}{Concepts}
    \setlength{\textwidth}{11.2cm}
    \begin{block}{Definition}
    \begin{table}
        \begin{tabular}{|m{3.2cm}|m{7cm}|}
            \hline
            Axiom & Meaning\\ 
            \hline
            5) Compatibility of scalarmultiplication & $a(bv)=(ab)v,a,b \in \mathbb{K}, v \in V$\\
            \hline
            6) Identity element of scalar multiplipication & $1v=v$, where 1 denotes the multiplicative identity in $\mathbb{K}$ \\
            \hline
            7)  Distributivity  of  scalar multiplication with respect to vector addition & $\alpha(u+v)=\alpha u+\alpha v, \alpha \in  $\mathbb{K}$ ,u,v \in V$\\
            \hline
            8)  Distributivity  of  scalar multiplication with respect to field addition & $(a+b)v=av+bv,a,b \in \mathbb{K}, v \in V$\\
            \hline
        \end{tabular}
        \label{tab:my_label}
    \end{table}
    \end{block}
\end{frame}
\begin{frame}{Vector space}
\setlength{\textwidth}{11.2cm}
    \begin{block}{Properties}
    Let V be a $\mathbb{R}$-vector space. Hence,
    \begin{enumerate}
        \item The zero vector (0) is unique
        \item The inverse vector of each vector $v$, denoted by $-v$, is unique.
        \item If $\lambda$v = 0, then 
            $\left\[
            \begin{array}{}
            \lambda =0\\
            v = 0 (The zero vector)
            \end{array}
            \right.$
    \end{enumerate}
    \end{block}
\end{frame}
\begin{frame}{Illustration of a vector space}
    \begin{columns}
    \column{0.65\textwidth}
    \includegraphics[height=7cm,width=6cm]{vector space.png}
    \column{0.35\textwidth}
    \begin{block}{Vector space}
    A vector space (also called a linear space) is a set of objects called vectors, which may be added together and multiplied ("scaled") by numbers, called scalars.
    \end{block}
    \end{columns}
\end{frame}
\section{Subspace}
\begin{frame}{Subspace}
\setlength{\textwidth}{11.2cm}
    \begin{block}{Definition}
    Let $(V,+,.)$ be a vector space over field $\mathbb{K}$ an $W\neq\varnothing$ be a subset of $V$. Now, $W$ is called a subspace of $V$ if $(W,+,.)$ is vector space. 
    \end{block}
    \begin{block}{Theorem}
    Let $(V,+,.)$ be a vector space over field $\mathbb{K}$ and $W\neq\varnothing$ be a subset of $V$ .Now,$W$ is called a subspace of $V$ if $W$ is closed under two operations in $V$, i.e.
    \begin{itemize}
        \item[i)] $\forall x,y \in W: x+y \in W$
        \item[ii)] $\forall x \in W, \forall k \in K: kx \in W$
    \end{itemize}
    \end{block}
    Remarks\\
    Two above conditions are equivalent to\\
    \begin{table}
    \centering
    \begin{tabular}{|c|}
         \hline
         $\forall x,y \in W, \forall k,l \in W: kx+ly \in W$\\
         \hline
    \end{tabular}
    \end{table}
\end{frame}
\end{document}
